\documentclass[letterpaper]{jpconf}
\usepackage{graphicx}
\usepackage{iopams}
\usepackage[pdftex,
      colorlinks=true,
      urlcolor=blue,       % \href{...}{...} external (URL)
      filecolor=green,     % \href{...} local file
      linkcolor=red,       % \ref{...} and \pageref{...}
      pagebackref,
      pdfpagemode=UseNone,
      bookmarksopen=true]{hyperref}


\begin{document}
% paper title
\title{Many-core applications to online track reconstruction in HEP experiments}
% Author list
\author{S.~Amerio$^1$, 
  D.~Bastieri$^1$, 
  M.~Corvo$^1$, 
  A.~Gianelle$^1$, 
  W.~Ketchum$^2$,
  T.~Liu$^3$, 
  A.~Lonardo$^4$, 
  D.~Lucchesi$^1$,
  S.~Poprocki$^5$, 
  R.~Rivera$^3$, 
  P.~Vicini$^4$
  and 
  P.~Wittich$^5$,
}
\address{$^1$ INFN and University of Padova, Italy}
\address{$^2$ Los Alamos National Laboratory, New Mexico, USA}
\address{$^3$ Fermi National Accelerator Laboratory, Illinois, USA}
\address{$^4$ INFN Roma, Italy}
\address{$^5$ Cornell University, New York, USA}

\ead{silvia.amerio@pd.infn.it}

\begin{abstract}
  %%% NSS PAPER abstract
  One of the most important issues that particle physics experiments
  at hadron colliders have to solve is real-time selection of the most
  interesting events. Typical collision frequencies do not allow all
  events to be written to tape for offline analysis, and in most
  cases, only a small fraction can be saved. The most commonly used
  strategy is based on two or three selection levels, with the low
  level ones usually exploiting dedicated hardware to decide within a
  few to ten microseconds if the event should be kept or not. This
  strict time requirement has made the usage of commercial devices
  inadequate, but recent improvements to Graphics Processing Units
  (GPUs) have substantially changed the conditions. Thanks to their
  highly parallel, multi-threaded, multicore architecture with
  remarkable computational power and high memory bandwidth, these
  commercial devices may be used in scientific applications, among
  which the event selection system (trigger) in particular may
  benefit, even at low levels. This paper describes the results of an
  R\&D project to study the performance of GPU technology for low
  latency applications, such as HEP fast tracking trigger algorithms.
  On two different setups, we measure the latency to transfer data
  to/from the GPU, exploring the timing of different I/O technologies
  on different GPU models. We then describe the implementation and the
  performance of a track fitting algorithm which mimics the CDF
  Silicon Vertex Tracker.  These studies provide performance
  benchmarks to investigate the potential and limitations of GPUs for
  future real-time applications in HEP experiments.
\end{abstract}

\section{Introduction}
%%% Verbatim from NSS paper
Real-time event reconstruction plays a fundamental role in High Energy
Physics (HEP) experiments at hadron colliders.  Reducing the rate of
data to be saved on tape from millions to hundreds of events per
second is critical. To increase the purity of the collected samples,
rate reduction has to be coupled with an initial selection of the most
interesting events.  In a typical hadron collider experiment, the
event rate has to be reduced from tens of MHz to a few kHz.  The
selection system (trigger) is usually organized in multiple levels,
each capable of performing a finer selection on more complex physics
objects describing the event. Trigger systems usually comprise a first
level based on custom hardware, followed by one or two levels usually
based on farms of general purpose processors.  The possibility of also
using commercial devices at a low trigger level is very appealing:
they are subject to continuous performance improvements driven by the
consumer market, are less expensive than dedicated hardware, and are
easier to support.  Among the commercial devices, Graphic Processing
Units (GPUs) are of particular interest for online selections given
their impressive computing power (the latest NVIDIA~\cite{bib_nvidia}
GPU architecture, Kepler, exceeds Teraflop computing power); moreover,
high-level programming architectures based on C/C++ such as
\textsc{cuda} and \textsc{opencl} make these devices more accessible
to the general user.  However, GPUs are not designed for low-latency
response, which is a critical requirement in a trigger system.

We showed initial studies on GPU performance in low-latency
environments ($\sim$100\,$\mu$s) in a previous paper~\cite{TIPP2011,
  NSS2012}.  In this paper we extend those studies measuring the
timing performance of different GPU cards in different data I/O
environments. The algorithm run on the GPU is a complete version of
the fast track fitting algorithm of the Silicon Vertex Tracker (SVT)
system at CDF~\cite{SVT1}.

The goal of this study is to investigate the strengths and weaknesses
of GPUs when applied in a low latency environment, with particular
emphasis on the data transfer latency to/from the GPU and the
algorithm latency for processing on the GPU in a manner similar to a
typical HEP trigger application.


\section{Experimental setup and data flow}
The studies are performed on two different experimental setups,
equipped with different data transfer links and GPU cards.  In both
setups two PCs are used, one acting as a transmitter (TX) and the
other as a receiver (RX).  Data are transferred from TX to RX,
processed on the GPU, and sent back to the receiver (see
Fig.~\ref{fig_data_flow}).  The latency for a complete loop is
measured on the transmitter using the time stamp counter register.

In this setup, the transmitter can represent the detector, as
the source of the data, or an upstream trigger processor, as
the ultimate sink of the data, while the the receiver is the
trigger system: the time to transfer data to the receiver is thus a
rough estimate of the latency to transfer the data from the detector
front-end to the trigger system.
 
The input data is a set of 32-bit words, each representing a set of
hits from the detector. When the track fitting algorithm is run, the
output data is composed of 32-bit words as well, representing the
results of the fit; when the input data is not processed by the
fitting algorithm, the output data is a .... 


%% gives unnumbered ack in CHEP style
\ack
The authors would like to thank the Fermilab staff and the FTK group at the 
University of Chicago for their support. This work was supported by the
U.S. Department of Energy, the U.S. National Science Foundation and the Italian
Istituto Nazionale di Fisica Nucleare. 


%% CHEP2013 recommended style for bibliography
\bibliographystyle{iopart-num}

\bibliography{gpu}

 
\end{document}


